\chapter{Unit testing \label{unittest}}

\section{Introduction \label{unittest-intro}}

Many aspects in the design flow of modern digital hardware design can
be viewed as a special kind of software development. From that
viewpoint, it is a natural question whether advances in software
design techniques can not also be applied to hardware design.

One software design approach that gets a lot of attention recently is
\index{extreme programming}%
\emph{Extreme Programming} (XP). It is a fascinating set of techniques and
guidelines that often seems to go against the conventional wisdom. On
other occasions, XP just seems to emphasize the common sense, which
doesn't always coincide with common practice. For example, XP stresses
the importance of normal workweeks, if we are to have the
fresh mind needed for good software development.

It is not my intention nor qualification to present a tutorial on
Extreme Programming. Instead, in this section I will highlight one XP
concept which I think is very relevant to hardware design: the
importance and methodology of unit testing.

\section{The importance of unit tests \label{unittest-why}}

Unit testing is one of the corner stones of Extreme Programming. Other
XP concepts, such as collective ownership of code and continuous
refinement, are only possible by having unit tests. Moreover, XP
emphasizes that writing unit tests should be automated, that they should
test everything in every class, and that they should run perfectly all
the time. 

I believe that these concepts apply directly to hardware design. In
addition, unit tests are a way to manage simulation time. For example,
a state machine that runs very slowly on infrequent events may be
impossible to verify at the system level, even on the fastest
simulator. On the other hand, it may be easy to verify it exhaustively
in a unit test, even on the slowest simulator.

It is clear that unit tests have compelling advantages. On the other
hand, if we need to test everything, we have to write
lots of unit tests. So it should be easy and pleasant
to create, manage and run them. Therefore, XP emphasizes the need for
a unit test framework that supports these tasks. In this chapter,
we will explore the use of the \code{unittest} module from
the standard Python library for creating unit tests for hardware
designs.


\section{Unit test development \label{unittest-dev}}

In this section, we will informally explore the application of unit
test techniques to hardware design. We will do so by a (small)
example: testing a binary to Gray encoder as introduced in
section~\ref{intro-indexing}. 

\subsection{Defining the requirements \label{unittest-req}}

We start by defining the requirements. For a Gray encoder, we want to
the output to comply with Gray code characteristics. Let's define a
\dfn{code} as a list of \dfn{codewords}, where a codeword is a bit
string. A code of order \code{n} has \code{2**n} codewords.

A well-known characteristic is the one that Gray codes are all about:

\newtheorem{reqGray}{Requirement}
\begin{reqGray} 
Consecutive codewords in a Gray code should differ in a single bit.
\end{reqGray}

Is this sufficient? Not quite: suppose for example that an
implementation returns the lsb of each binary input. This would comply
with the requirement, but is obviously not what we want. Also, we don't
want the bit width of Gray codewords to exceed the bit width of the
binary codewords.

\begin{reqGray} 
Each codeword in a Gray code of order n must occur exactly once in the
binary code of the same order.
\end{reqGray}

With the requirements written down we can proceed.

\subsection{Writing the test first \label{unittest-first}}

A fascinating guideline in the XP world is to write the unit test
first. That is, before implementing something, first write the test
that will verify it. This seems to go against our natural inclination,
and certainly against common practices. Many engineers like to
implement first and think about verification afterwards.

But if you think about it, it makes a lot of sense to deal with
verification first. Verification is about the requirements only --- so
your thoughts are not yet cluttered with implementation details. The
unit tests are an executable description of the requirements, so they
will be better understood and it will be very clear what needs to be
done. Consequently, the implementation should go smoother. Perhaps
most importantly, the test is available when you are done
implementing, and can be run anytime by anybody to verify changes.

Python has a standard \code{unittest} module that facilitates writing,
managing and running unit tests. With \code{unittest}, a test case is 
written by creating a class that inherits from
\code{unittest.TestCase}. Individual tests are created by methods of
that class: all method names that start with \code{test} are
considered to be tests of the test case.

We will define a test case for the Gray code properties, and then
write a test for each of the requirements. The outline of the test
case class is as follows:

\begin{verbatim}
from unittest import TestCase

class TestGrayCodeProperties(TestCase):

    def testSingleBitChange(self):
     """ Check that only one bit changes in successive codewords """
     ....


    def testUniqueCodeWords(self):
        """ Check that all codewords occur exactly once """
    ....
\end{verbatim}

Each method will be a small test bench that tests a single
requirement. To write the tests, we don't need an implementation of
the Gray encoder, but we do need the interface of the design. We can
specify this by a dummy implementation, as follows:

\begin{verbatim}
def bin2gray(B, G, width):
    ### NOT IMPLEMENTED YET! ###
    yield None
\end{verbatim}

For the first requirement, we will write a test bench that applies all
consecutive input numbers, and compares the current output with the
previous one for each input. Then we check that the difference is a
single bit. We will test all Gray codes up to a certain order
\code{MAX_WIDTH}.

\begin{verbatim}
    def testSingleBitChange(self):
        """ Check that only one bit changes in successive codewords """
        
        def test(B, G, width):
            B.next = intbv(0)
            yield delay(10)
            for i in range(1, 2**width):
                G_Z.next = G
                B.next = intbv(i)
                yield delay(10)
                diffcode = bin(G ^ G_Z)
                self.assertEqual(diffcode.count('1'), 1)
        
        for width in range(1, MAX_WIDTH):
            B = Signal(intbv(-1))
            G = Signal(intbv(0))
            G_Z = Signal(intbv(0))
            dut = bin2gray(B, G, width)
            check = test(B, G, width)
            sim = Simulation(dut, check)
            sim.run(quiet=1)
\end{verbatim}

Note how the actual check is performed by a \code{self.assertEqual}
method, defined by the \code{unittest.TestCase} class.

Similarly, we write a test bench for the second requirement. Again, we
simulate all numbers, and put the result in a list. The requirement
implies that if we sort the result list, we should get a range of
numbers:

\begin{verbatim}
    def testUniqueCodeWords(self):
        """ Check that all codewords occur exactly once """

        def test(B, G, width):
            actual = []
            for i in range(2**width):
                B.next = intbv(i)
                yield delay(10)
                actual.append(int(G))
            actual.sort()
            expected = range(2**width)
            self.assertEqual(actual, expected)
       
        for width in range(1, MAX_WIDTH):
            B = Signal(intbv(-1))
            G = Signal(intbv(0))
            dut = bin2gray(B, G, width)
            check = test(B, G, width)
            sim = Simulation(dut, check)
            sim.run(quiet=1)
\end{verbatim}


\subsection{Test-driven implementation \label{unittest-impl}}

With the test written, we begin with the implementation. For
illustration purposes, we will intentionally write some incorrect
implementations to see how the test behaves.

The easiest way to run tests defined with the \code{unittest}
framework, is to put a call to its \code{main} method at the end of
the test module:

\begin{verbatim}
unittest.main()
\end{verbatim}

Let's run the test using the dummy Gray encoder shown earlier:

\begin{verbatim}
% python test_gray.py -v
Check that only one bit changes in successive codewords ... FAIL
Check that all codewords occur exactly once ... FAIL
<trace backs not shown>
\end{verbatim}

As expected, this fails completely. Let us try an incorrect
implementation, that puts the lsb of in the input on the output:

\begin{verbatim}
def bin2gray(B, G, width):
    ### INCORRECT - DEMO PURPOSE ONLY! ###

    @always_comb
    def logic():
        G.next = B[0]

    return logic
\end{verbatim}


Running the test produces:

\begin{verbatim}
% python test_gray.py -v
Check that only one bit changes in successive codewords ... ok
Check that all codewords occur exactly once ... FAIL

======================================================================
FAIL: Check that all codewords occur exactly once
----------------------------------------------------------------------
Traceback (most recent call last):
  File "test_gray.py", line 109, in testUniqueCodeWords
    sim.run(quiet=1)
...
  File "test_gray.py", line 104, in test
    self.assertEqual(actual, expected)
  File "/usr/local/lib/python2.2/unittest.py", line 286, in failUnlessEqual
    raise self.failureException, \
AssertionError: [0, 0, 1, 1] != [0, 1, 2, 3]

----------------------------------------------------------------------
Ran 2 tests in 0.785s
\end{verbatim}

Now the test passes the first requirement, as expected, but fails the
second one. After the test feedback, a full traceback is shown that
can help to debug the test output.

Finally, if we use the correct implementation as in
section~\ref{intro-indexing}, the output is:

\begin{verbatim}
% python test_gray.py -v
Check that only one bit changes in successive codewords ... ok
Check that all codewords occur exactly once ... ok

----------------------------------------------------------------------
Ran 2 tests in 6.364s

OK
\end{verbatim}



\subsection{Changing requirements \label{unittest-change}}

In the previous section, we concentrated on the general requirements
of a Gray code. It is possible to specify these without specifying the
actual code. It is easy to see that there are several codes
that satisfy these requirements. In good XP style, we only tested
the requirements and nothing more.

It may be that more control is needed. For example, the requirement
may be for a particular code, instead of compliance with general
properties. As an illustration, we will show how to test for
\emph{the} original Gray code, which is one specific instance that
satisfies the requirements of the previous section. In this particular
case, this test will actually be easier than the previous one.

We denote the original Gray code of order \code{n} as \code{Ln}. Some
examples: 

\begin{verbatim}
L1 = ['0', '1']
L2 = ['00', '01', '11', '10']
L3 = ['000', '001', '011', '010', '110', '111', '101', 100']
\end{verbatim}

It is possible to specify these codes by a recursive algorithm, as
follows:

\begin{enumerate}
\item L1 = ['0', '1']
\item Ln+1 can be obtained from Ln as follows. Create a new code Ln0 by
prefixing all codewords of Ln with '0'. Create another new code Ln1 by
prefixing all codewords of Ln with '1', and reversing their
order. Ln+1 is the concatenation of Ln0 and Ln1.
\end{enumerate}

Python is well-known for its elegant algorithmic
descriptions, and this is a good example. We can write the algorithm
in Python as follows:

\begin{verbatim}
def nextLn(Ln):
    """ Return Gray code Ln+1, given Ln. """
    Ln0 = ['0' + codeword for codeword in Ln]
    Ln1 = ['1' + codeword for codeword in Ln]
    Ln1.reverse()
    return Ln0 + Ln1
\end{verbatim}

The code \samp{['0' + codeword for ...]} is called a \dfn{list
comprehension}. It is a concise way to describe lists built by short
computations in a for loop.

The requirement is now that the output code matches the
expected code Ln. We use the \code{nextLn} function to compute the
expected result. The new test case code is as follows:

\begin{verbatim}
class TestOriginalGrayCode(TestCase):

    def testOriginalGrayCode(self):
        """ Check that the code is an original Gray code """

        Rn = []
        
        def stimulus(B, G, n):
            for i in range(2**n):
                B.next = intbv(i)
                yield delay(10)
                Rn.append(bin(G, width=n))
        
        Ln = ['0', '1'] # n == 1
        for n in range(2, MAX_WIDTH):
            Ln = nextLn(Ln)
            del Rn[:]
            B = Signal(intbv(-1))
            G = Signal(intbv(0))
            dut = bin2gray(B, G, n)
            stim = stimulus(B, G, n)
            sim = Simulation(dut, stim)
            sim.run(quiet=1)
            self.assertEqual(Ln, Rn)
\end{verbatim}

As it happens, our implementation is apparently an original Gray code:

\begin{verbatim}
% python test_gray.py -v TestOriginalGrayCode
Check that the code is an original Gray code ... ok

----------------------------------------------------------------------
Ran 1 tests in 3.091s

OK
\end{verbatim}


 

