\documentclass{article}
\usepackage[utf8]{inputenc}
\usepackage{amsmath}
\usepackage{mathtools}
\inputencoding{utf8}

\title{Hypothesis 2: Every index on a stock exchange can be presented as a combination of several other indexes and some of these variables have a postponed impact (Regression models with lags).}

\begin{document}
\maketitle

\section{General Idea}
Here is general idea broken into several steps is described.
\begin{enumerate}
	\item Main Index is chosen -\textgreater X
	\item The number of possible indexes is almost not limited - only impossible or weakly correlated markets are excluded. Chosen indexes and corresponding time-series are saved to an array for further use.
	\item For the main index X "best-match"-regression will be found through iteration over a number of correlated indexes. The model will be manually limited to N indexes. The combination of these indexes should result in the closest trend in comparison to the original trend of the index X. \\
	\[X = f(x_{i_{t_{j}}}),\thinspace where \thinspace i=\overline{0,n},\thinspace and \thinspace j=\overline{-T,0} \]
	\[T, n \in N \]    
	\item For these indexes: \[x_{i_{t_{j}}}\] forecasts with probability will be taken (here we do not work directly with texts, because if first hypothese works, that will allow step text analysis over. At least in this Bachelor work this tusk is an excessive task).
	\item Afterwards all these forecasts will be integrated in one general forecast for main Index X
\end{enumerate}
This hypothesis now aims not only already existing companies, listed on stock exchanges, but also can be useful for new-products, know-hows and IPOs, where indirect information can be usefull for proper estimation.


\newpage
\section{Trivial and non-trivial sub-tasks}
Probable problems, bottle-neck places and trivial tasks of the work.
\subsection{Trivial tasks}
Trivial sub-tasks are steps that are easy to implement, they are normally transparent and run-time is not an import issue.
\begin{enumerate}
	\item The step with the text processing is excessive.
	\item Here buzz and mood can be used as "non-explicit" indicators for fine-tunning.
	\item Since we are not working with texts, but with forecasts directly we will most likely find \[[probability,\thinspace forecast]_{i_{t_{j}}}\] without a problem.
	
\end{enumerate}  

\subsection{Non-trivial tasks}
Here most obvious problems of the implementation are announced.

\begin{enumerate}
	\item The firts problem is to choose number of the indexes we will later iterate over in order to find the closest "representatives". Here not only cross-murket correlation should be considered, but also time-dependance on these correlations.
	\item The task of finding the optimum is already hard enough. Here proper objective function needs to be found and also some additional error-estimators.
	\item Iterating through list of indexes should also contain restriction on crosscorelation among those indexes.
	\item Integration and interpretation of the forecasts will be hard and not transparent
	\item Run-time and complexity, scaling - all these can be problematic.  
\end{enumerate}

\newpage
\section {Conclusion}
This hyopthesis aims not only existing companies listetd on the stock exchange, but also brand-new companies, know-hows and IPOs - the cases, wehere people will most definetly (under-) overestimate the company, because of lack of the information.\\
This hypothesis is deepen version of the hypothesis 1. First hypothesis can be use as a starting test: if there are any connections.
\begin{enumerate}
	\item Positive case\\
In positive case second "loop" (hypothesis 2) checks for all most possible dependencies and gives additional inforamtion about specific company. This work can be potentially implemented as a bigger module for StockPulse.
	\item Negative case\\
In this case it will be hard to say why it didn't work: 
\begin{enumerate}
	\item Small list for iteration?
	\item Wrong model?
	\item The case of the new-product is indeed unique and nobody can properly estimate (fuel-, technic-, medicine-breakthroughs)?
	\item First Hypothesis was wrong.
\end{enumerate}
\end{enumerate}

\end{document}