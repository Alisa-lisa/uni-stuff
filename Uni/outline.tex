\documentclass {article}
\usepackage[utf8]{inputenc}
\usepackage{hyperref}
\usepackage{color}


\title{Bachelor 2015}
\author{Alisa Dammer}

\begin{document}
\maketitle
 
\begin{enumerate}
	\item[1.] Introduction:
		\begin{enumerate}
			\item[1.1] Motivation
			\item[1.2] Overview of the existing approaches, models and programs
			\item[1.3] Explanation of the Thesis of the Bachelor
				\begin{enumerate}
					\item[1.3.1] Theoretical background and the problem statement
					\item[1.3.2] The program broken into several steps
						\begin{enumerate}
							\item[1.3.2.1] Data preparation
							\item[1.3.2.2] Model fine-tuning
							\item[1.3.2.3] Checking the hypothesis for specified limitations
						\end{enumerate}
				\end{enumerate} 
		\end{enumerate}
	\item[2.] The model
		\begin{enumerate}
			\item[2.1] Data preparation (correlation matrices and further filtering)
			\item[2.2] Model specification and limitations
				\begin{enumerate}
					\item[2.2.1] Specification of the quality and quantity of the parameters
					\item[2.2.2] Iterative fine-tuning of the model 
					\item[2.2.3] Error check
				\end{enumerate}
			\item[2.3] Second data processing (sentimental analysis for the found indexes)
			\item[2.4] Integration of the sub-results into final result
			\item[2.5] Testing of the result
		\end{enumerate}
	\item[3.] Conclusion
		\begin{enumerate}
			\item[3.1] The interpretation of the results
			\item[3.2] Bottle neck and other problems of the model
			\item[3.3] The potential and further use
			\item[3.4] Possible improvements
		\end{enumerate}
	\item[4.] References
\end{enumerate}

\newpage
\section{Introduction}
\subsection{Motivation}
Big Data\\
Data mining\\
Sentiment data analysis
\subsection{An overview of the existing approaches, models and programs}
classical econometric - regressions\\
Among the classical econometric approaches two main directions can be called: linear and non-linear regressions. First of all we will consider linear regressions and discuss their advantages and disadvantages.\\
According to the most common definition, in statistics, linear regression is an approach for modeling the relationship between a scalar dependent variable y and explanatory variable denoted X [1]. There are one-variable regressions called simple linear regressions and multiple regressions, which shows the dependencies of the dependent variable from more than one explanatory variables. The regression equation looks following way:\\
\[Y = a + bX + \varepsilon\]
where Y - dependent variable, X - explanatory variable, a - the intercept of regression line, b - the slope of the regression line, $\varepsilon$ - the error term.\\
The simple linear regression is based on 6 strict assumptions:
\begin{enumerate}
	\item Because of the linear form neither parameter a, nor parameter b may have higher power than one.
	\item The independent variable X is not random.
	\item The variance is constant. Which means that the expected error value is 0.
	\item The variance of $\varepsilon$ is constant for all observations.
	\item Errors are not correlated.
	\item The error's distribution is normal.
\end{enumerate}
On one hand the simple linear regression has several disadvantages: the form of the equation (here number of the predicting variables is concerned) and underlying assumptions. The dependency from only one variable can not describe fully all the relationship on the market. Even if you consider an equations system, consisting from linear regressions with different variables, it will lead to unnecessary complexity and not realistic or complete dependencies between variables. The second disadvantage is the assumptions - they are unrealistic in modern world and are appropriate only for very simplified model of he world.\\
The big advantage of the simple linear regression on the other hand is it's estimators: ordinary lest squares, generalized least squares, instrumental variables, maximum likelihood (ML) and various ML techniques - these are the most known and not too complex approaches to estimate the regression. There also many other different techniques, but we won't discuss them here.\\ 
Since it is likely to be impossible to build a proper prediction for the variable only with one predicting variable, multiple linear regression was invented. As you can see from the name, multiple linear regression describes relationship between more than one explanatory variables and dependent variable. The form of the equation looks as following:\[ Y = a + b_{1}X_{1}+ ... + b_{n}X_{n}, \thinspace \thinspace where \thinspace \thinspace n \in N\]   
$b_{i}$ - the "contribution" of the i-th variable to the regression, a -the intercept of the regression line, Y - dependant variable, X - explanatory variables.\\
Unlike simple linear regression, multiple linear regression does not have the form-disadvantage. However, the linear nature of the regression provides the model with several assumptions:
\begin{enumerate}
	\item All $b_{i}$ have the potential of 1 (are linear).
	\item The residuals are distributed normally. (Residuals - the difference between predicted value and observed value).
	\item Uncertain number of the explanatory variables X - there is no technique to choose exact number of the variables that will optimally predict the value Y.
	\item Completely "substitutive" variables X - unnecessary big number of the variables without increasing the accuracy of the prediction.
	\item Complex form of the variables (X can have a higher-order polynomial form) leads to reduction of transparency and wrong results. 
	  
\end{enumerate}
machine learning\\
neural networks\\
Most successful approaches: Top 5
\subsection{Explanation of the Thesis of the Bachelor}
The idea: "how to check the hypothesis" - in several logical steps.

\newpage
\section{The model}
\subsection{Data preparation}
Here independent of the main index (not chosen yet) based on correlation-matrices certain amount of indexes will be taken for further consideration. For example every index with correlation-vector bigger or equal to +-0.3\\
Some period of the time-series (for all indexes) will be left as test-sample. (let's say about a month-period).\\
All time series will be cleaned from the noise using following technics: mooving average, extracting a trend line on high frequency long-term data, 
\subsection{Model specification and limitations}
First, the equation or the model description will be given, then the number for the parameters will be set (with all necessary explanations).\\
Second, the automatic optimum finding for given limitations will be maid. (python iterative optimum-finding test). At the end another target function, built from the found optimum will be tested (error check).\\
Third, texts for the chosen indexes will be analysed. The forecast, probability and maybe buzz will be the results of the sub-chapter.\\
Forth, the results of the previous stage will be united into one forecast and probability for the main index (index on the left side of the main equation).\\
Finally, the results of the model will be compared with the actual results that we left as the test-sample.

\newpage
\section{conclusion}
\subsection{Interpretation of the results}
Here the difference of the actual value (test-sample) and our forecasts will be discussed and explained.
\subsection{Problems}
Here the biggest difficulties will be stated and also discussed the influence of the strong limitations.
\subsection{Potential}
Here most possible use and advantage of the model will be discussed.
\subsection{Improvements}
In order to be more useful the program can take several improvements...\\
Theoretical\\
Data\\
Technical

\newpage
\section{References}
\begin{enumerate}
	\item Wikipedia: linear regeression \url{http://en.wikipedia.org/wiki/Linear_regression}
\end{enumerate}

\end{document}