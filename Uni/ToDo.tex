\documentclass {article}
\usepackage[utf8]{inputenc}
\usepackage{mathtools}
\usepackage{pifont}
\usepackage{color}
\usepackage{hyperref}


\begin{document}
ToDo:
\begin{enumerate}
	\item Week 1: 5.01-9.01.2105 - Overview of the models 
	\begin{enumerate}
		\item[5.01.2015]: Classical and modern econometric models:  linear regressions (Trading volume importance, ) \ding{52}
		\item[6.01.2015]: Simple linear regression 	 \ding{52}		  
		\item[7.01.2015]: Multiple linear regression \ding{52} 
		\item[8.01.2015]: Nonlinear regressions (simple and multiple) print [non linear d, e] \ding{52} 
		\item[9.01.2015]: Neural networks (MLP, DBM, BMs, )\ding{52}
		\item[10.01.2015]: Machine learning on stock exchange as generalization for different approaches (deep learning algorithm as "THE TREND")\ding{52} 
	\end{enumerate}
	\item Week 2: 12.01-22.01.2105 - Data preprocessing and collection
	\begin{enumerate}
		\item[12.01.2015]: Pause \ding{52} 
		\item[13.01.2015]: Finishing first week, Model choosing, Intro first-iteration finishing (data mining and machine learning. Choosing the model) \ding{52} 
		\item[19.01.2015]: Data collection  \& Data-analysis (auto-regression, heteroskedasticity, error and data-distribution, density of the distribution)\ding{52}
		\item[20.01.2015]: Data collection \ding{52} 
		Smoothing the input data (moving average, noise reduction techniques)
		\item[21.01.2015]: Dependencies matrix  (time-dependent correlations and non-time dependent correlations)
		\item[22.01.2015]: Cut off (general dependencies on the main index)
		
	\end{enumerate}
	\item Week 3: 5.02-8.02.2105 - Model training and fine-tuning
	\begin{enumerate}
		\item[5.02.2015]: Optimum finding  \[ R^2 -> max \] as example of the focus functions
		\item[6.02.2015]: Iterative fine-tuning and learning process 1
		\item[7.02.2015]: Iterative fine-tuning and learning process 2
		\item[8.02.2015]: Error check
	\end{enumerate}
	\item Week 4: 9.02-15.02.2105 - Second Data processing
	\begin{enumerate}
		\item[9.02.2015]: Texts mining (manual evaluation + automized mood detection, python library Open source) 
		\item[10.02.2015]: Texts mining (manual evaluation + automized mood detection, python library Open source)
		\item[11.02.2015]: Manual analysis direct(later automatization) and automatic sentiment analysis
		\item[12.02.2015]: Manual analysis direct(later automatization) and automatic sentiment analysis
		\item[13.02.2015]: Forecast as time series
		\item[14.02.2015]: Forecast as a time series $->$ Input for 1 stage
		\item[15.02.2015]: Fine-tune the regression
	\end{enumerate}
	\item Week 5: 16.02-22.02.2105 - Results Interpretation and clean up
	\begin{enumerate}
		\item[16.02.2015]: Fine-tuning the regression model
		\item[17.02.2015]: Comparison with the original parameters
		\item[18.02.2015]: Validation of the model
		\item[19.02.2015]: Interpretation of the results
		\item[20.02.2015]: Text clean up (in intro: parametric and non-parametric regressions, give a definition of the regressions, give histograms to prove that distribution restrictions are too strict, examples with real statistics from R)
		\item[21.02.2015]: Clean up, further automatization.
	\end{enumerate}
\end{enumerate}

\end{document}