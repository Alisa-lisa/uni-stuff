\documentclass {article}
\usepackage[utf8]{inputenc}
\usepackage{mathtools}
\usepackage{pifont}


\begin{document}
ToDo:
\begin{enumerate}
	\item Week 1: 5.01-9.01.2105 - Overview of the models 
	\begin{enumerate}
		\item[5.01.2015]: Classical and modern econometric models:  linear regressions (Trading volume importance, ) \ding{52}
		\item[6.01.2015]: Simple linear regression 	 \ding{52}		  
		\item[7.01.2015]: Multiple linear regression \ding{52} 
		\item[8.01.2015]: Nonlinear regressions (simple and multiple) print [non linear d, e] \ding{52} 
		\item[9.01.2015]: Neural networks (MLP, DBM, BMs, )
		\item[10.01.2015]: Machine learning on stock exchange as generalization fro different approaches (deep learning algorithm as "THE TREND")
	\end{enumerate}
	\item Week 2: 12.01-16.01.2105 - Data preprocessing
	\begin{enumerate}
		\item[12.01.2015]: Data collection 
		\item[13.01.2015]: Data-analysis (autoregression, heteroskedasticity, error and data-distribution, density of the distribution)
		\item[14.01.2015]: Smoothing the input data (moving average, noise reduction techniques)
		\item[15.01.2015]: Dependencies matrix  (time-dependent correlations and non-time dependent correations)
		\item[16.01.2015]: Cut off (general dependencies on the main index)
		\item[17.01.2015]: Cut off (cross-dependencies, decide what to do with the substitutes)
	\end{enumerate}
	\item Week 3: 19.01-24.01.2105 - Model training and fine-tuning
	\begin{enumerate}
		\item[12.01.2015]: Optimum finding  \[ R^2 -> max \] as example of the focus functions
		\item[13.01.2015]: Iterative fine-tuning and learning process 1
		\item[14.01.2015]: Iterative fine-tuning and learning process 2
		\item[15.01.2015]: Iterative fine-tuning and learning process 3
		\item[16.01.2015]: Error check
	\end{enumerate}
	\item Week 4: 26.01-30.01.2105 - Second Data processing
	\begin{enumerate}
		\item[12.01.2015]: Texts mining (manual evaluation) 
		\item[13.01.2015]: Texts mining (manual evaluation)
		\item[14.01.2015]: Manual analysis direct(later automatization) and automatic sentiment analysis
		\item[15.01.2015]: Manual analysis direct(later automatization) and automatic sentiment analysis
		\item[16.01.2015]: Results interpretation, Integration 
	\end{enumerate}
	\item Week 5: 2.02-6.02.2105 - Results Interpretation and clean up
	\begin{enumerate}
		\item[2.02.2015]: Test -> on test sample
		\item[3.02.2015]: Test -> real values vs predicted values 
		\item[4.02.2015]: Interpretation
		\item[5.02.2015]: Inerpretation
		\item[6.02.2015]: Text clean up (in intro: parametric and non-parametric regressions, give a definition of the regressions, give histograms to prove that distribution restrictions are too strict, examples with real statistics from R)
	\end{enumerate}
\end{enumerate}

\end{document}