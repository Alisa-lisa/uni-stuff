\documentclass{article}
\usepackage[utf8]{inputenc}
\usepackage{color}
\usepackage{mathtools}
\usepackage{color}
\usepackage{hyperref}
\hypersetup{urlbordercolor={1 1 1}}


\begin{document}
The Structure of the thesis
\begin{enumerate}
    \item[1.] Motivation
        \begin{enumerate}
            \item[1.1] Python language overview
            \item[1.2] R language overview
            \item[1.3] Python vs R wars
            \item[1.4] Thesis description
            \item[1.5] Data preparation 
        \end{enumerate}
    \item[2.] Python
        \begin{enumerate}
            \item[2.1] The structure of the program
            \item[2.2] The process description
            \begin{enumerate}
                \item[2.2.1] Used Libraries
                \item[2.2.2] Documentation quality and quantity
                \item[2.2.3] Advantages and disadvantages of the program 
            \end{enumerate}
            \item[2.3] Results
        \end{enumerate}
    \item[3.] R
        \begin{enumerate}
            \item[3.1] The structure of the program
            \item[3.2] The process description
            \begin{enumerate}
                \item[3.2.1] Used Libraries
                \item[3.2.2] Documentation quality and quantity
                \item[3.2.3] Advantages and disadvantages of the program 
            \end{enumerate}
            \item[3.3] Results
        \end{enumerate}
    \item[4.] Results
        \begin{enumerate}
            \item[1.1] Objective comparison
            \item[1.2] Subjective comparison
            \item[1.3] Conclusion
        \end{enumerate}
    \item[5.] Literature
\end{enumerate}

\newpage
\section{Motivation}
Every time requires it's own heroes. Nowadays, with a massive digitalization of the world, statistical analysis is not only used in spheres like classical production and distribution (logistics), but also everywhere on the web. In order to create a nice web-site, that will attract buyers and more important will make them purchase things from a particular on-line shop using this exact site, on-line shops require an analysis of a customer's behavior, specific demands and preferences. Another example where data analysis is essential is video games industry. The AI, which is built in an immense data source and is constantly analyzing the incoming data from the players, servers and third party sources (if needed), is used in many game-projects from a simple browser-strategy to "the most expected game of the year". The game play, the enemies, the load distribution on the servers and many other things are based on analysis and forecast. The are many-many other examples where data analysis is used.\vspace{3mm}\\
Since the computer technologies are way more advanced, than they were several decades ago, the computations can be automatized, the data collection can be automatized. For every purposes certain tools, languages and libraries should be chosen carefully among the great choice. In this work we will concentrate our attention on a statistic-econometric task. On the whole, almost every language can be used to build a program for collecting, processing and visualizing data: SCALA, C, C++, .NET, Java, Python, JavaScript, R, etc.. However, some of the languages are more suitable for this purposes.\vspace{3mm}\\
In this work we will concentrate on "easy" languages. We will compare Python 3 (Python 2 in form of some libraries) language with R language. Both of them are relatively new (Python is about 25 years old and R is about 22 years old) to the wide publicity. Both are currently used for data analysis (for Python it is not the only use, but as we said before, we will concentrate on this aspect). Both languages are widely used in open source projects and have a large community behind them. Also, very important that both of the languages are relatively easy to begin with also for non-programmers. Python has a gradual learning curve due to its simplicity, clear and intuitive syntax. Although R has a steep learning curve it is still pretty easy to use for small and not too complex projects (which is the case for this thesis).

\newpage
\subsection{Python language overview}
The Python language is pretty young. It has first appeared about 25 years ago. As described on the official page in Wikipedia:"Python is a general-purpose, high-level programming language"[1]. That means that Python can be used for many tasks like back end development, front end development, data analysis and even for processor simulation (the python code will generate you VHDL code eventually). The language supports multiple programming paradigms: Object Oriented, Imperative and Functional programming.\\
Another big advantage of the language is being a cross-platform language executable via interpreter that can be installed on every operation system. Another possibility to run python programs without installing an interpreter is to package the program into stand-alone executable.\\ 
The language has an automatic garbage collector, so you don't have to worry about memory leaks by small to medium programs.\\
Enormous community stays behind Python and its \textcolor{blue}{variations}. The standard library has more than enough functions for easy start. The language's syntax is very clear and intuitive in comparison to C++, C or even Java. I would say the language is similar to normal human speech but in more strict and logical form.\\
Another good thing about Python is build-in test - doctest module. This test is very easy to use to check and debug your code on the fly. The use of the module is intuitively clear even to beginners. You also don't keep the test's text in other file, which gives you certain level of an overview of the program. Unit test are also widely used among the programmers using Python language. There also other possibilities to debug and benchmark the code written in python, but we will not discuss it here.\\
The disadvantage of the language is its speed in comparison with C, C++ and Java. Also the visualization tools could be better, but since the language is a multipurpose language, it is normal, that some things are not top among the class. \\

\subsection{R language overview}
According to Wikipedia:"R is a programming language and software environment for statistical computing and graphics"[5]. R is a successor of the S language, purposed about 39 years ago. R is distributed under GNU General Public license, which makes this language a nice choice for an open source analytical project.\\
Same as Python, R also uses interpreter to execute the code. R language is pretty much single-purpose language. The main use of R is statistical analysis and visualization. Since R is an open source project, there is a huge \textcolor{blue}{community of real life practice and scientists behind it}.\\
The disadvantage of the language is its one-purpose nature. Analytical project built with R will work with no doubt very fast and good, presenting good results. But the use of such project will be reduced, since there are very few quick and comfortable ways to integrate the program into the system.\\

\subsection{Python vs R wars}
Since both languages offer tons of packages and libraries for analyzing and visualization, people argue about what language is a better fit. You can find a lot of discussions about this topic on stackoverflow [6], stackoverflow analog for data science [7] and many other resources on internet [8],[9],[10].
So far nobody has managed to give a satisfactory final answer, since the languages indeed are different. However for every concrete task and scale we can give a list of requirements and compare the languages objectively leaving the subjective comparison aside.\\
Tho formal criteria to compare Python and R are:
\begin{enumerate}
    \item The amount of useful resources for the task
    \item Clear documentation with examples
    \item Performance
    \item Memory use
    \item Appropriate data structures
    \item Possibility to work with Big Data
    \item Visualization tools
    \item Hard or soft limitations
    \item Need of workarounds
\end{enumerate}
Using this list both languages can be objectively evaluated. These points are appropriate for all small-, middle- and big-projects. In next section we will describe the test-task to compare Python and R languages and define the points-system for objective evaluation.\\

\subsection{Thesis description}
The idea of this bachelor thesis is to test which language suits better specific statistic task. 
The program that will be evaluated consists of four parts:
\begin{enumerate}
    \item Getting and formatting data - the program should be able to read a prepared file in csv format. Afterwards, the program should format the data into needed structure for further use.
    \item Analyzing data - the program should be able to run statistical test to figure out data characteristics. This step is needed to determine what regression models are allowed for this data set. Following test will be presented: stationary test, build cdf and kde, find moments, distribution test (goodness of fit tests).
    \item Building a model - the program should be able to create a proper model using a step-forward algorithm, set the limits on the number of the predictors and return the names of the parameters, regression parameters and some useful statistics.
    \item Visualization - the program should be able to present the results and steps between if needed.
\end{enumerate}
The program will build a simple regression if possible. This task is considered to be middle-size task, not too easy but at the same time not too time and knowledge consuming.
The data will be prepared and provided as 2 files (training set and test set) in .csv format.\\
According to the list for objective evaluation, each point will give either 0 or 1 score to language. The 0 score will be given in negative case and score 1 will be given in positive case.
\begin{enumerate}
    \item[] Useful resources - at least 2 different libraries for one task gives 1 point.
    \item[] Documentation - examples, clean source code and clear structured APIs give 1 point.
    \item[] Performance - fastest language gets 1 point
    \item[] Memory - the program with the smallest memory use gets 1 point.
    \item[] Data structures - no additional formatting needed gives 1 point.
    \item[] Big Data - libraries, plugins and frameworks for big data give 1 point.
    \item[] Visualization - the least amount of time spent on this task gives 1 point.
    \item[] Limitations - if there are any limitations, the language gets 0 points.
    \item[] Workarounds - additional time to solve the task gives 0 points (deviation from the time for the naive implementation).
\end{enumerate}
The results of the programs will be also compared. If they differ, the models will be cross-compared wit the other program or manually.\\
The task is to build a best matching linear regression (if possible) for a company (always the first column in the file) using certain limitations on the number of the predictors.

\subsection{Data preparation}
As was mentioned before the 


\newpage
\section{Python}
\subsection{The structure of the program}
\subsection{The process description}
\subsubsection{Used Libraries}
\subsubsection{Documentation quality and quantity}
\subsubsection{Advantages and disadvantages of the program}
\subsection{Results}

\newpage
\section{R}
\subsection{The structure of the program}
\subsection{The process description}
\subsubsection{Used Libraries}
\subsubsection{Documentation quality and quantity}
\subsubsection{Advantages and disadvantages of the program}
\subsection{Results}

\newpage
\section{Results}
\subsection{Objective comparison}
\subsection{Subjective comparison}
\subsection{Conclusion}

\newpage
\section{Literature}
\begin{enumerate}
    \item[1] \url{https://en.wikipedia.org/wiki/Python_(programming_language)}
    \item[2]
    \item[3] \url{https://www.r-project.org/about.html}
    \item[4] \url{http://www.revolutionanalytics.com/what-r}
    \item[5] \url{https://en.wikipedia.org/wiki/R_(programming_language)}
    \item[6] \url{http://stackoverflow.com/questions/2770030/r-or-python-for-file-manipulation}
    \item[7] \url{http://datascience.stackexchange.com/questions/326/python-vs-r-for-machine-learning}
    \item[8] \url{http://www.kdnuggets.com/2015/05/r-vs-python-data-science.html}
    \item[9] \url{http://blog.datacamp.com/r-or-python-for-data-analysis/}
    \item[10] \url{http://101.datascience.community/2015/05/12/data-science-wars-r-vs-python/}
    \item[11]
\end{enumerate}
\end{document}