\documentclass{article}
\usepackage[utf8]{inputenc}
\usepackage{color}
\usepackage{mathtools}
\usepackage{color}


\begin{document}
The Structure of the thesis
\begin{enumerate}
    \item[1.] Motivation
        \begin{enumerate}
            \item[1.1] Python language overview
            \item[1.2] R language overview
            \item[1.3] Python vs R wars
            \item[1.4] Thesis description
        \end{enumerate}
    \item[2.] Python
        \begin{enumerate}
            \item[2.1] The structure of the program
            \item[2.2] The process description
            \begin{enumerate}
                \item[2.2.1] Used Libraries
                \item[2.2.2] Documentation quality and quantity
                \item[2.2.3] Advantages and disadvantages of the program 
            \end{enumerate}
            \item[2.3] Results
        \end{enumerate}
    \item[3.] R
        \begin{enumerate}
            \item[3.1] The structure of the program
            \item[3.2] The process description
            \begin{enumerate}
                \item[3.2.1] Used Libraries
                \item[3.2.2] Documentation quality and quantity
                \item[3.2.3] Advantages and disadvantages of the program 
            \end{enumerate}
            \item[3.3] Results
        \end{enumerate}
    \item[4.] Results
        \begin{enumerate}
            \item[1.1] Objective comparison
            \item[1.2] Subjective comparison
            \item[1.3] Conclusion
        \end{enumerate}
    \item[5.] Literature
\end{enumerate}

\newpage
\section{Motivation}
Every time requires it's own heroes. Nowadays, with a massive digitalization of the world, statistical analysis is not only used in spheres like classical production and distribution (logistics), but also everywhere on the web. On-line shops requires an analysis of a customer's behavior, specific demands and preferences to create a nice web-site, that will attract buyers and more important will make them purchase special goods on a certain site. Another application of data analysis is, of course, video games. The AI used in many game-projects from simple browser-strategy to "most expected of the year" is built on immense data source and its constantly analyzing the incoming data from the players, servers and third party sources (if needed). The game play, the enemy, the load distribution on the servers and many other things are based on statistics and forecast. The are many-many other examples where data analysis is used.\vspace{3mm}\\
Since the computer technologies are way more advanced, than they were several decades ago, the computations can be automatized, the data collection can be automatized. For that purposes certain tools, languages and libraries should be chosen. On the whole, almost every language can be used to build a program for collecting, processing and visualizing data, like: SCALA, C, C++, .NET, Java, Python, JavaScript, R, etc.. However, some of the languages are more suitable for some purposes.\vspace{3mm}\\
In this work we will compare Python 3 language with R language. Both of them are relatively new \textcolor{blue}{(How old are they)} to the wide publicity. Both are used for data analysis (and not only, but this aspect interests us the most). Both languages are widely used for open source projects and have a large community behind them. Also very important that the learning curve for both languages is Exponential \textcolor{blue}{(is it so?)}.
 
\subsection{Python language overview}
The Python language is pretty young. It has first appeared about 24 years ago. As described on the official page in Wikipedia, Python is a general-purpose, high-level programming language. That means that Python can be used for many tasks like back end development, front end development, data analysis and even for processor simulation instead of VHDL or Verilog. This multi-use is possible, since the language support multiple programming paradigms: Object Oriented, Imperative and Functional programming.\\
Another big advantage of the language is being a cross-platform language executable via interpreter that can be installed on every operation system. Another possibility to run python programs without installing an interpreter is to package the program into stand-alone executable.\\ 
The language has an automatic garbage collector, so you don't have to worry about memory leaks by small to medium programs.\\
Enormous community stays behind Python and its variations. The standard library has more than enough functions to start. The language's syntax is very easy in comparison to C++, C or even Java. I would say the language is similar to normal human speech but in more strict and logical form.s\\
Another good thing about Python is build-in test - doctest module. This test is very easy to use to check and debug your code on the fly. The use of the module is intuitively clear even to beginners. You also don't keep the test's text in other file, which gives you certain level of an overview of the program. Unit test are also widely used among the programmers using Python language. 
\subsection{R language overview}

\subsection{Python vs R wars}

\subsection{Thesis description}


\newpage
\section{Python}
\subsection{The structure of the program}
\subsection{The process description}
\subsubsection{Used Libraries}
\subsubsection{Documentation quality and quantity}
\subsubsection{Advantages and disadvantages of the program}
\subsection{Results}

\newpage
\section{R}
\subsection{The structure of the program}
\subsection{The process description}
\subsubsection{Used Libraries}
\subsubsection{Documentation quality and quantity}
\subsubsection{Advantages and disadvantages of the program}
\subsection{Results}

\newpage
\section{Results}
\subsection{Objective comparison}
\subsection{Subjective comparison}
\subsection{Conclusion}

\newpage
\section{Literature}

\end{document}