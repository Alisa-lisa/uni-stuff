\documentclass{article}
\usepackage[utf8]{inputenc}



\begin{document}
\S238 HGB: Buchführungspflicht\\
(1) Jeder Kaufmann ist verpflichtet, Bücher zu führen und in diesen seine Handelsgeschäfte und die Lage seines Vermögens nach den Grundsätzen ordnungsmäßiger Buchführung ersichtlich zu machen. Die Buchführung muß so beschaffen sein, daß sie einem sachverständigen Dritten innerhalb angemessener Zeit einen Überblick über die Geschäftsvorfälle und über die Lage des Unternehmens vermitteln kann. Die Geschäftsvorfälle müssen sich in ihrer Entstehung und Abwicklung verfolgen lassen.\\
(2) Der Kaufmann ist verpflichtet, eine mit der Urschrift übereinstimmende Wiedergabe der abgesandten Handelsbriefe (Kopie, Abdruck, Abschrift oder sonstige Wiedergabe des Wortlauts auf einem Schrift-, Bild- oder anderen Datenträger) zurückzubehalten.\\
\S140 AO: Buchführungs- und Aufzeichnungspflichten nach anderen Gesetzen\\
Wer nach anderen Gesetzen als den Steuergesetzen Bücher und Aufzeichnungen zu führen hat, die für die Besteuerung von Bedeutung sind, hat die Verpflichtungen, die ihm nach den anderen Gesetzen obliegen, auch für die Besteuerung zu erfüllen.\\
\S141 AO: Buchführungspflicht bestimmter Steuerpflichtiger\\
(1) Gewerbliche Unternehmer sowie Land- und Forstwirte, die nach den Feststellungen der Finanzbehörde für den einzelnen Betrieb
\begin{enumerate}
	\item[1.]Umsätze einschließlich der steuerfreien Umsätze, ausgenommen die Umsätze nach § 4 Nr. 8 bis 10 des Umsatzsteuergesetzes, von mehr als 500 000 Euro im Kalenderjahr oder
	\item[2.](weggefallen)
	\item[3.]selbstbewirtschaftete land- und forstwirtschaftliche Flächen mit einem Wirtschaftswert (§ 46 des Bewertungsgesetzes) von mehr als 25 000 Euro oder
	\item[4.]einen Gewinn aus Gewerbebetrieb von mehr als 50 000 Euro im Wirtschaftsjahr oder
	\item[5.]einen Gewinn aus Land- und Forstwirtschaft von mehr als 50 000 Euro im Kalenderjahr
\end{enumerate}
gehabt haben, sind auch dann verpflichtet, für diesen Betrieb Bücher zu führen und auf Grund jährlicher Bestandsaufnahmen Abschlüsse zu machen, wenn sich eine Buchführungspflicht nicht aus § 140 ergibt. Die §§ 238, 240, 241, 242 Abs. 1 und die §§ 243 bis 256 des Handelsgesetzbuchs gelten sinngemäß, sofern sich nicht aus den Steuergesetzen etwas anderes ergibt. Bei der Anwendung der Nummer 3 ist der Wirtschaftswert aller vom Land- und Forstwirt selbstbewirtschafteten Flächen maßgebend, unabhängig davon, ob sie in seinem Eigentum stehen oder nicht.

(2) Die Verpflichtung nach Absatz 1 ist vom Beginn des Wirtschaftsjahrs an zu erfüllen, das auf die Bekanntgabe der Mitteilung folgt, durch die die Finanzbehörde auf den Beginn dieser Verpflichtung hingewiesen hat. Die Verpflichtung endet mit dem Ablauf des Wirtschaftsjahrs, das auf das Wirtschaftsjahr folgt, in dem die Finanzbehörde feststellt, dass die Voraussetzungen nach Absatz 1 nicht mehr vorliegen.

(3) Die Buchführungspflicht geht auf denjenigen über, der den Betrieb im Ganzen zur Bewirtschaftung als Eigentümer oder Nutzungsberechtigter übernimmt. Ein Hinweis nach Absatz 2 auf den Beginn der Buchführungspflicht ist nicht erforderlich.

(4) Absatz 1 Nr. 5 in der vorstehenden Fassung ist erstmals auf den Gewinn des Kalenderjahrs 1980 anzuwenden.

\end{document}