\documentclass[ngerman]{gdb-aufgabenblatt}
\usepackage{paralist}

\renewcommand{\Aufgabenblatt}{4}
\renewcommand{\Ausgabedatum}{Mi. 27.11.2013}
\renewcommand{\Abgabedatum}{Do. 12.12.2013}
\renewcommand{\Gruppe}{Dammer, Teuteberg, Wilhelm}


\begin{document}

\section{Relationenalgebra}

\begin{compactenum}[(a)]
	\item Welche Obstsorten wurden von einem Entdecker mit Vorname Horst entdeckt? 
	\begin{align*}
	 &\projektion{Sorte}(Obst\verbund{Entdecker=PNR} \selektion{Vorname=\wert{Horst}}(Person))
	\\  
\end{align*}\
	\item 
	\begin{align*}
 	&\projektion{Vorname,Nachname}(Person\verbund{PNR=Person} \selektion{Symptom=\wert{Halskratzen}}(Allergie))
	\\  
\end{align*}\
	\item 
	\begin{align*}
 	\projektion{Sorte,Nachname}(\projektion{Person,Sorte}(\selektion{Person=Entdecker}(Obst\verbund{ONR=Obst} \selektion{Symptom=\wert{Wurgreiz}}(Allergie)))\\
 	\verbund{Person=PNR}(\projektion{PNR,Nachname}(Person))
	\\  
\end{align*}\
\end{compactenum}






\section{SQL-Schemadefinition}
\begin{enumerate}
	\item[a)] 
	\begin{verbatim}
	CREATE TABLE Rennfahrer 
	(RID           INT           PRIMARY KEY,
	 Vorname       CHAR(30)      NOT NULL,
	 Nachname      CHAR(30)      NOT NULL,
	 Geburt        DATE          NOT NULL,
	 Wohnort       CHAR(50)      NOT NULL,
	 CONSTRAINT    Rennstall   FOREIGN KEY(Rennstall) 
	 REFERNCES Rennstall(RSID));
	 
	CREATE TABLE Rennstall
	(RID           INT           PRIMARY KEY,
	 Name          CHAR(30)      UNIQUE NOT NULL,
	 Teamchef      CHAR(30)      UNIQUE NOT NULL,
	 Budget        INT           NOT NULL,
	 CONSTRAINT Budget CHECK(0 < Budget < 500));
	 
	CREATE TABLE Rennort
	(OID           INT           PRIMARY KEY,
	 Name          CHAR(30)      UNIQUE NOT NULL, 
	 Strecke       CAHR(50)      UNIQUE NOT NULL);
	   
	CREATE TABLE Platzierung
	(RID           INT           NOT NULL,           
	 OID           INT           NOT NULL,
	 Platz         INT           NOT NULL,
	 CONSTRAINT RID FOREIGN KEY(Rennfahrer) REFERNCES Rennfahrer(RID),
	 CONSTRAINT OID FOREIGN KEY(Rennort) REFERNCES Rennort(OID));
	   	   	
	\end{verbatim}
	\item[b)] NO ACTION und RESTRICT sind dasselbe Operationen\\
			  NO ACTION - jeder Versuch, einen Primärschlüssel zu löschen oder zu ändern, unterbunden wird, wenn es dazu einen Fremschlüsselwert in der referenzierten Tabelle gibt.\\
			  RESTRICT - weist die Lösch- oder Änderungsoperation auf der Elterntabelle zurück.\\
	          Resrtictions:Zusammengehörige Spalten im Fremdschlüssel und referenzierten Schlüssel müssen ähnliche Datentypen haben, damit sie sich ohne Typkonvertierung vergleichen lassen. Die Größe und das Vorzeichen von Integer-Typen müssen gleich sein. Die Länge von String-Typen muss nicht unbedingt identisch sein. Wenn Sie SET NULL verlangen, dürfen die Spalten der Kindtabelle nicht als NOT NULL deklariert sein.\\
	          
	          
	\item[c)] INSERT INTO Rennfahrer VALUES\\
			  (8, "Sebastian", "Vettel", 1987-07-03, "Kemmental (Schweiz)", 2),\\
			  (6, "Fernando", "Alonso", 1981-07-29, "Lugano (Schweiz)", 5),\\
			  (8, "Marc", "Webber", 1976-08-27, "Aston Clinton (UK)", 2),\\
			  (9, "Lewis", "Hamilton", 1985-01-07, "Genf (Schweiz)", 31),\\
			  (20, "Jenson", "Button", 1980-01-19, "Monte Carlo (Monaco)", 31),\\
			  (21, "Felipe", "Massa", 1982-04-25, "S\"ao Paulo (Brazilien)", 5),\\
			  (44, "Kimi", "R\"aikk\"onen", 1979-10-17, "Espoo (Finnland)", 34);\\
			  
			  INSERT INTO Rennstall VALUES\\
			  (2, "Red Bull", "Christian Horner", 370),\\
			  (5, "Ferrari", "Stefano Domenicali", 350),\\
			  (31, "McLaren", "Martin Whitmarsh", 220),\\
			  (34, "Lotus F1", "Eric Boullier", 100);\\
			  
			  INSERT INTO Rennort VALUES\\
			  (4, "Australian GP", "Albert Park Circuit"),\\
			  (15, "Malaysia GP", "Sepang International Circuit"),\\
			  (34, "China GP", "Shanghai International Circuit");\\
			  
			  INSERT INTO Platzierung VALUES\\
			  (8, 4, 6),\\
			  (4, 15, 1),\\
			  (20, 15, 17),\\
			  (4, 4, 3),\\
			  (6, 4, 2),\\
			  (8, 15, 2),\\
			  (6, 21, 1),\\
			  (9, 4, 5),\\
			  (21, 15, 5),\\
			  (20, 4, 9),\\
			  (21, 4, 4);\\
			     
	\item[d.1)] DELETE FROM Rennfahrer WHERE Vorname LIKE F\%
	\item[d.2)] DROP TABLE Platzierung\\
				DROP TABLE Rennort\\
	            DROP TABLE Rennstall\\
		        DROP TABLE Rennfahrer
		
\end{enumerate} 


\newpage
\section{SQL-Anfragen}

\begin{compactenum}[(a)]
\item \begin{verbatim}
SELECT DISTINCT
  o.Sorte
FROM 
  Obst o,
  Person p,
  Allergie a
WHERE
  p.Vorname = 'Peter' AND
  p.Nachname = 'Meyer' AND
  a.Person = p.PNR AND
  a.Obst = o.ONR
ORDER BY
  o.Sorte DESC
\end{verbatim}
\item \begin{verbatim}
SELECT
  p.PNR,
  p.Nachname,
  COUNT(p.PNR)
FROM 
  Person p,
  Allergie a
WHERE
  p.PNR = a.Person
GROUP BY
  p.PNR
\end{verbatim}
\item \begin{verbatim}
SELECT
  p.PNR
FROM 
  Person p,
  Obst o
WHERE
  o.Entdecker = p.PNR
GROUP BY
  p.PNR
HAVING
  COUNT(p.PNR) > 6
\end{verbatim}
\item \begin{verbatim}
SELECT
  p.Vorname,
  p.Nachname
FROM 
  Person p,
  Person q,
  Obst o
WHERE
  p.Lieblingsobst = o.ONR AND
  o.Entdecker = q.PNR AND
  p.Vorname = q.Vorname
\end{verbatim}
\item \begin{verbatim}
SELECT
  p.PNR,
  p.Vorname,
  p.Nachname
FROM 
  Person p
WHERE
  p.PNR NOT IN
  {SELECT 
     o.Entdecker
   FROM
     Obst o}
\end{verbatim}

\end{compactenum}


\section{Optimierung}
\begin{compactenum}[(a)]
	\item Operatorbaum f\"ur die vorgegebene Anfrage:
\end{compactenum}	
\begin{tikzpicture}
\node (Obst) {Obst};
\node (Person) [left=25mm of Obst] {Person};
\node (join1) [above=20mm of $(Obst)!.5!(Person)$] {$\verbund{Lieblingsobst=ONR}$};
\node (selektion1) [above=of join1] {$\selektion{Sorte=\wert{K\%}}$};
\node (projektion) [above=of selektion1] {$\projektion{PNR,Vorname,Nachname}$};
\node (final) [above=of projektion] {};

\path (Obst) edge node[smallr,near start,above right] {25 Tupel\\3 Attribute} (join1);
\path (Person) edge node[smalll,near start,above left] {2000 Tupel\\5 Attribute} (join1);
\path (join1) edge node[smallr,near start,above left] {2000 Tupel\\8 Attribute} (selektion1);
\path (selektion1) edge node[smallr,midway,left] {$2000\cdot\frac{5}{25}=400$ Tupel\\8 Attribute} (projektion);
\path (projektion) edge node[smallr,midway,left] {$400$ Tupel\\3 Attribut} (final);
\end{tikzpicture}
\newpage
\begin{compactenum}[(b)]
\item Optimierung:
\end{compactenum}

\begin{tikzpicture}
\node (Obst) {Obst};
\node (Person) [left=25mm of Obst] {Person};
\node (selektion1) [above=of Obst] {$\selektion{Sorte=\wert{K\%}}$};

\node (join1) [above=25mm of $(selektion1)!.5!(Person)$] {$\verbund{Lieblingsobst=ONR}$};
\node (projektion1) [above=of join1] {$\projektion{PNR,Vorname,Nachname}$};
\node (final) [above=of projektion1] {};

\path (Obst) edge node[smallr,near start,above right] {25 Tupel\\3 Attribute} (selektion1);
\path (Person) edge node[smalll,near start,above left] {2000 Tupel\\5 Attribute} (join1);
\path (selektion1) edge node[smallr,near start,right] {$25\cdot\frac{5}{25}=5$ Tupel\\3 Attribute} (join1);

\path(join1) edge node[smallr,near start,above left] {$2000\cdot\frac{1}{5}=400$Tupel\\ 8 Attribute} (projektion1);
\path (projektion1) edge node[smallr,midway,left] {$400$ Tupel\\3 Attribut} (final);
\end{tikzpicture}








\end{document}