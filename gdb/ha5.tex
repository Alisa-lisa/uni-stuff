\documentclass[a4paper,12pt]{scrartcl}

\usepackage[utf8]{inputenc}
\usepackage[ngerman]{babel}
\usepackage{scrpage2}\pagestyle{scrheadings}
\usepackage{graphicx}
\usepackage{pgfplots}
\usepackage{multicol}
\usepackage{tikz}
\usepackage{amssymb}
\usepackage{amsmath}
\usepackage{ulem}

\ihead{Aufgabenblatt 5}
\ohead{\today}
\chead{Gruppe Dammer, Teuteberg, Wilhelm}
\pagestyle{scrheadings}

\begin{document}

\section{Aufgabe 1}
* L\"osung *

\section{Aufgabe 2}
*L\"osung*

\section{Aufgabe 3} 	
*L\"osung*

\section{Aufgabe 4} 	
\begin{enumerate}
\item[a)] {B}; {A,D}

\item[b)] {C}; {E}

\item[c)]
Zuerst, präfen wir 1NF:\\
"Jedes Attribut der Relation muss einen atomaren Wertebereich haben, und die Relation muss frei von Wiederholungsgruppen sein".(Wiki)\\
Die Relation befindet sich im {\itshape \large 1NF}, da die geschachtelte Wertebereiche enthält.\\

Jetzt prüfen wir 2NF:\\
"Eine Relation ist in der zweiten Normalform, wenn die erste Normalform vorliegt und kein Nichtschlüsselattribut funktional abhängig von einer echten Teilmenge eines Schlüsselkandidaten ist".(Wiki)\\
Hier Menge {C, E} ist von Menge {B, AD} funktional abhängig. D.h. R befindet sich nicht in 2NF.\\ 
Antwort: Die realation befindet sich in der 1NF

\end{enumerate}

\end{document}